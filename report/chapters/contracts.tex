\chapter{Verification of the \texorpdfstring{\msp}{MSP430} ISA}

The first contribution of this thesis is a mechanized proof of the security properties established by IPE. The verification process involves:
\begin{enumerate}
\item codifying IPE as a Sail specification;
\item translating such specification to Rocq to be used with Katamaran;
\item stating the desired guarantees as universal contracts, and
\item proving them with the help of Katamaran's symbolic executor.
\end{enumerate}

Due to the vulnerabilities covered in \cref{sec:ipe-attacks}, it would be impossible to successfully verify IPE's security on a faithful model of the \msp; the inteded security guarantees do not hold there. Rather, we will work on a specification that fixes the faults in IPE's design that are exploited in the attacks. Our work should then tell us that there are no additional undiscovered vulnerabilities.

\section{Formalizing \texorpdfstring{\msp}{MSP430}'s ISA}

Texas Instruments doesn't provide a precise, machine-readable description of the \msp's ISA, which we need for our proof. Hence, the first step is to develop a formal specification in Sail, starting from the official user manuals.

The intention is to model the common behavior of the family described in \cite{slau367p}. In some places the specification relies on some model-specific detail (\eg the addresses the MPU registers are mapped to); we refer to the \msp[FR5969]'s manual \cite{slas704g} in such cases. Another source of information about IPE and the bootcode is \cite{slaa685}.

The development of a Sail model of the \msp was already started in \cite{mspthesis}, with a number of limitations, including a restricted set of supported addressing modes, lack of interrupt and reset handling and no support for the Memory Protection Unit and IPE. In \cref{sec:sail-ipe} we show how the model is extended with IPE, and in \cref{sec:model-limitations} we discuss the effect of the missing features.

\subsection{Modeling IPE}
\label{sec:sail-ipe}

Extending the model to support IPE requires adding the configuration registers and inserting permission checks every time memory is accessed.

\subsubsection{MPU registers}

The IPE configuration registers (listed in \cref{tab:ipe-registers}) are part of the MPU register file, which we model as a whole, even though we don't have full support for the MPU.

\begin{table}
  \centering
  \begin{tabular}{ll} \toprule
    Register & Function \\ \midrule
    \reg{MPUIPC0} & control register: enabled (\regbit{MPUIPENA}, bit 6), locked (\regbit{MPUIPLOCK}, bit 7) \\
    \reg{MPUIPSEGB2} & upper boundary of IPE region* (inclusive) \\
    \reg{MPUIPSEGB1} & lower boundary of IPE region* (exclusive) \\ \bottomrule
  \end{tabular}
  *without the 4 least significant bits
  \caption{IPE registers.}
  \label{tab:ipe-registers}
\end{table}

For each register we define a Sail \sail{register} (essentially a global variable) and a variant in the enum \sail{mpu_register_name} to refer to it symbolically. The MPU register file is mapped to memory starting at address \sail{mpu_reg_base} (\addr{050a} in the \msp[fr5969]), so we also record the offset (in machine words) of each register's location relative to \sail{mpu_reg_base} (\cref{lst:ipe-registers-definition}) so that we can convert between addresses and register names later.

\begin{lstlisting}[
  language=sail,
  float, label=lst:ipe-registers-definition,
  caption={IPE register declarations (other MPU registers omitted).}
]
  register MPUIPC0_reg    : wordBits
  register MPUIPSEGB2_reg : wordBits
  register MPUIPSEGB1_reg : wordBits

  enum mpu_register_name =
    { MPUIPC0, MPUIPSEGB2, MPUIPSEGB1, /* ... */ }

  mapping mpu_register_index : mpu_register_name <-> range(0, 7) =
    { MPUIPC0 <-> 5, MPUIPSEGB2 <-> 6, MPUIPSEGB1 <-> 7, /* ... */ }
\end{lstlisting}

To enable reading and writing these registers we hook into the existing \sail{readMem} and \sail{writeMem} functions, that are wrappers around the foreign functions \sail{read_ram} and \sail{write_ram} (for the latter see \cref{sec:sail-structure}). If the address that is being accessed belongs to the MPU register file, the access is handled by \sail{read_mpu_reg_byte}/\sail{write_mpu_reg_byte}, otherwise by \sail{readMem}/\sail{writeMem} (\cref{lst:if-mpu-reg}).

\begin{lstlisting}[
  language=sail,
  float, label=lst:if-mpu-reg,
  caption={Extract from \sail{readMem}.} % actually read_mem_aux
]
  if is_mpu_reg_addr(addr)
  then read_mpu_reg_byte(addr)
  else read_ram(/* ... */, addr)
\end{lstlisting}

\Cref{lst:write_mpu_reg_byte} shows how writing to the registers is implemented. The offset \sail{idx} of the accessed address from \sail{mpu_reg_base} is used to identify the register (through the \sail{mpu_register_index} mapping from \cref{lst:ipe-registers-definition}). If the IPE registers are locked (\ie bit \regbit{MPUIPLOCK} of \reg{MPUIPC0} is set) the write has no effect. Memory is byte-addressable and registers are 16 bit wide, so \sail{write_mpu_reg_byte} distinguishes between writing to either byte of the register.

\begin{lstlisting}[
  language=sail,
  float, label=lst:write_mpu_reg_byte,
  caption={\sail{write_mpu_reg_byte} (parts related to MPU omitted).}
]
  val write_mpu_reg_byte : (Address, byteBits) -> unit
  function write_mpu_reg_byte(addr, v) =
    let idx : int = unsigned(addr >> 1) - unsigned(mpu_reg_base >> 1) in
    let low_byte = addr[0..0] == 0b0 in
    if idx >= 0 & idx < 8 then
      let reg = mpu_register_index(idx) in

      // ...

      // prevent write on locked registers
      else if (// ...
            | (is_ipe_reg(reg) & MPUIPC0_reg[MPUIPLOCK_bit..MPUIPLOCK_bit] == 0b1)
      then ()

      else if low_byte
      then match reg {
        // ...
        MPUIPC0    => MPUIPC0_reg[7..0] = v,
        MPUIPSEGB2 => MPUIPSEGB2_reg[7..0] = v,
        MPUIPSEGB1 => MPUIPSEGB1_reg[7..0] = v,
      }
      else match reg {
        // ...
        MPUIPC0    => MPUIPC0_reg[15..8] = v,
        MPUIPSEGB2 => MPUIPSEGB2_reg[15..8] = v,
        MPUIPSEGB1 => MPUIPSEGB1_reg[15..8] = v,
      }

    else assert(false)
\end{lstlisting}

\subsubsection{Access permission check}

We need to ensure that all accesses to memory are allowed by the IPE configuration. First, we define a helper \sail{check_ipe_access} (\cref{lst:check_ipe_access}) that, given an address, returns true if access to that address is allowed, \ie:
\begin{itemize}
\item IPE is disabled (bit \regbit{MPUIPENA} of \reg{MPUIPC0} is unset) \circref{cia-enabled}, or
\item the address is unprotected \circref{cia-unprotected}, \ie outside of the region delimited by \sail{MPUIPSEGB1 << 4} and \sail{MPUIPSEGB2 << 4} (see the note in \cref{tab:ipe-registers}), or
\item we are executing trusted code \circref{cia-trusted}, \ie the PC is inside the IPE region except the first 8 bytes (reserved for the IPE configuration structure, see \cref{sec:bootcode}).
\end{itemize}

\startcstep
\begin{lstlisting}[
  language=sail, mathescape=true,
  float, label=lst:check_ipe_access,
  caption={Definition of \sail{check_ipe_access}. The \sail{jump} parameter and condition \circref{cia-jump} are discussed in \cref{sec:entry-point}.}
]
  val check_ipe_access : (Address, bool) -> bool
  function check_ipe_access(addr, jump) =
    let pc = unsigned(PC_reg) in
    MPUIPC0_reg[MPUIPENA_bit..MPUIPENA_bit] == 0b0 $\circdef{cia-enabled}$
    | ~(in_ipe_segment(addr)) $\circdef{cia-unprotected}$
    | (ipe_lower() + 8 <= pc & pc < ipe_higher()) $\circdef{cia-trusted}$
    | (jump & unsigned(addr) == ipe_lower() + 8) $\circdef{cia-jump}$
\end{lstlisting}

We again hook into \sail{readMem} and \sail{writeMem} to call \sail{check_ipe_access} and throw an \sail{ipe_violation} exception if it returns false.

\subsubsection{Restricted entry into the IPE region}
\label{sec:entry-point}

IPE as designed and implemented is susceptible to code reuse attacks (\cref{sec:code-reuse}). In our model we instead define a single entry point, the ninth byte in the IPE region---just after the space dedicated to the IPE configuration structure (\cref{sec:bootcode}). Changing the PC to point to a protected address constitutes an IPE violation if the code performing the change is untrusted (\ie the previous value of PC was outside the protected region).

An advantage of the entry point not being the first protected address is that it's not possible to accidentally ``fall through'' to the IPE region when executing the instruction immediatly preceding the lower boundary.

Multiple entry points can be emulated with a dispatcher as shown in \cref{lst:dispatch}, where an identifier for the trusted function to call is wrote to \reg{R4}.

\begin{lstlisting}[
  language=asm,
  float, label=lst:dispatch,
  caption={Emulation of multiple entry points via a dispatcher function.}
]
  untrusted:
      mov #1, r4
      call ipe_entry

  ipe_entry:
      cmp r4, #0
      beq trusted_fun_a
      cmp r4, #1
      beq trusted_fun_b
      ...
\end{lstlisting}

To implement this check, whenever the PC is modified we call \sail{check_ipe_access} with the \texttt{jump} parameter set to \sail{true} (\cref{lst:check_ipe_access}). The check then passes only if the code performing the jump has access to the target address \circref{cia-enabled}\circref{cia-unprotected}\circref{cia-trusted}, or if the target is the IPE entry point \circref{cia-jump}. There is no function that is called sistematically to modify registers (unlike \sail{writeMem} for memory), and Sail doesn't support implicit setters, so we need to carefully cover every change of PC. This includes direct assignments to the \sail{PC_reg} \sail{register} and the \sail{writeReg} function, that writes to a register given its symbolic name (a variant of the enum \sail{Register}).

\subsubsection{Ensuring proper PC modification order}
\label{sec:pc-mod-order}

If the same instruction can both modify the PC and access memory, the change to the PC must happen last to avoid attacks such as controlled call corruption (\cref{sec:call-corruption}).

The Sail specification already did this properly for \asm{call}, but not for instructions with two operands where the source register uses the \intro{indirect autoincrement} mode and the destination's mode (absolute or indexed) involves writing to memory. The instruction \asm{mov @pc+, &addr} reads the value stored at the address contained in the \reg{PC} register, increments \reg{PC} by 2, and writes the value to the address \sail{addr}. If the starting PC is untrusted and \sail{addr} is protected the operation should fail, even if \(\mathrm{PC} + 2\) is in the IPE region, so the permission check has to be performed before the increment---but it didn't in the existing Sail specification.

In practice this wouldn't have caused any problem, since it is not possible to reach the entry point by adding 2 to a PC outside the IPE region, so an IPE violation would be generated anyway.  % also the first 8 bytes don't grant privileges

There is no way to modify PC and access memory in the same instruction other than these two instructions. (TODO double-check)

\subsection{Limitations of the model}
\label{sec:model-limitations}

As anticipated, even after adding IPE the model is still not complete, but we believe this simplified specification to still be a worthwhile verification target.

The missing addressing modes add immediate parameters and absolute addresses, that give no more power to the attacker compared to the supported modes, which already allow read/write access to arbitrary memory locations.

The MPU is used to further \emph{restrict} access to memory, by dividing it in up to three segments with different permissions. It is intended mostly to prevent accidental modification of memory and not as a security feature, but can be combined with IPE to get stronger guarantees \cite{Bognar2024}. Regardless, \emph{not} including it restricts the attacker \emph{less}, so our verification doesn't lose any value from this.

Resets can be part of an exploit, as they disable register locking. We can exclude this possibility by making sure that IPE gets enabled and locked after every boot; this is a property we can prove about the device's bootcode.

The most significant omission is interrupt handling, which is an essential component of the attack sketched in \cref{sec:interrupts}. ...

\section{Translation to \texorpdfstring{\usail}{μSail}}

Katamaran works with \usail, a subset of Sail embedded in Rocq. We were able to generate a \usail version of the specification using the conversion tool provided by Katamaran\footnote{\url{https://github.com/katamaran-project/sail-backend/}}, after removing uses of features not supported by the core calculus.

In particolar, we needed to:
\begin{itemize}
\item Remove uses of the vector type and bitfields. Vectors are homogeneous static arrays; they were used to represent the MPU register file (instead of a number of different \sail{register}s) to simplify the implementation of unaligned word accesses. Bitfields are a convenience feature that allows naming certain bits and ranges in a bitvector for easier reference.

\item Remove all uses of polymorphic functions, substituting them with monomorphic equivalents. Polymorphism is not supported by \usail.

\item Split up large functions that pattern match on many arguments into several smaller functions. \usail doesn't support pattern matching on multiple values at once and or-patterns, so such match expressions become nested after translation. \Cref{lst:match-explosion} shows an example. In some cases this led to an explosion in the size of a \usail function, making it excessively slow to type-check in Rocq.

\item Declare the signatures for the foreign functions and indicate at each of their uses that the call is to a foreign function.

\item Perform some minor edits, such as adding missing type annotations for some statements.
\end{itemize}

\begin{lstlisting}[
  language=sail,
  float, label=lst:match-explosion,
  caption={Example of a multi-argument match and its translation.}
]
  // original: 4 arms           // translation: 9 arms
  match (x, y, z) {             match x {
    (X1, Y1, Z1) => e1,           X1 => match y {
    (X1, Y2, Z1) => e2,             Y1 => match z { Z1 => e1, _ => e4 },
    (X2, Y1, Z2) => e3,             Y2 => match z { Z1 => e2, _ => e4 },
    _            => e4              _  => e4,
  }                               },
                                  X2 => match y {
                                    Y1 => match z { Z2 => e3, _ => e4 },
                                    _  => e4,
                                  },
                                  _ => e4
                                }
\end{lstlisting}

After translation we end up with two Rocq files, \file{Base.v} and \file{Machine.v}. \file{Base.v} contains Rocq versions of the Sail user-defined types, see \cref{lst:coq-ast} for an example. We are also required to define here our ``memory model'', \ie a type that represents memory. For our purpose we don't need anything more than a function mapping each address to the value it stores:
\begin{minted}{coq}
  Definition Memory := Address -> byteBits.
\end{minted}
but more complex representations could be used to model features such as memory-mapped IO.

\begin{listing}
  \begin{minted}{coq}
    Inductive ast : Set :=
    | DOUBLEOP          : doubleop -> BW -> Register -> AM -> Register -> AM -> ast
    | SINGLEOP          : singleop -> BW -> AM -> Register -> ast
    | JUMP              : jump -> bv 10 -> ast
    | DOESNOTUNDERSTAND : bv 16 -> ast.
  \end{minted}
  \caption{Rocq translation of the Sail type representing a decoded instruction.}
  \label{lst:coq-ast}
\end{listing}

In \file{Machine.v} we find the \usail specification proper. See \cref{lst:translation-example} for an example of a \usail function definition contained in \file{Machine.v}. Here we also declare the foreign functions and lemmas used in the specification. % TODO maybe put them here?

\begin{listing}
  \begin{lstlisting}[language=sail]
    val setPC: wordBits -> unit
    function setPC(v) = writeReg(WORD_INSTRUCTION, PC, Word(v))
  \end{lstlisting}
  \begin{minted}{coq}
    Definition fun_setPC : Stm ["v" :: ty.bvec (16)] ty.unit :=
      stm_let "ga#205"
        (ty.union Uwordbyte)
        (stm_exp (exp_union Uwordbyte Kword (exp_var "v")))
        (stm_call writeReg
           (env.snoc
              (env.snoc
                 (env.snoc
                    (env.nil)
                    (_::_) (exp_val (ty.enum Ebw) WORD_INSTRUCTION)%exp)
                 (_::_) (exp_val (ty.enum Eregister) PC)%exp)
              (_::_) (exp_var "ga#205")%exp)).
  \end{minted}
  \caption{Code of a Sail function and its \usail translation (manually formatted).}
  \label{lst:translation-example}
\end{listing}

\section{IPE security as universal contract}

Now that we have a model of the \msp in a suitable format, we are ready to express the desired security guarantees by constructing a universal contract, as discussed in \cref{sec:universal-contracts}. We first need to decide on the exact property to prove, and then express it in Katamaran's separation logic.

\subsection{Target properties}
\label{sec:target-properties}

Our contract will focus on guaranteeing integrity and confidentiality of the IPE region when untrusted code is executed, and assuming that IPE is properly configured (\ie enabled, locked, \(\reg{MPUIPSEGB1} < \reg{MPUIPSEGB2}\)). Concretely, we want to ensure that:
\begin{itemize}
\item protected locations cannot be read and modified;
\item jumps into the IPE region are only allowed if the target is the designated entry point;
\item the IPE registers don't change their value.
\end{itemize}

It can then be proved separately that the bootcode sets the IPE registers to what is expected by the contract. We are not interested in verifying the security of trusted code, since it always has access to all memory; the only meaningful property we could prove is that it doesn't modify locked registers.

We consider a location protected if it lies within the IPE boundaries; there is no reason to have a special case for the IVT (see \cref{sec:ipe}) considering that we don't support interrupts.

``Untrusted code'' is all code that should not have access to protected memory. In addition to instructions not contained in the IPE region, the first 8 protected bytes also don't count as trusted. That said, we will treat them as such given that it simplifies the proofs and doesn't meaningfully weaken our result. The contract applies only to untrusted code, so this change does make it less applicable in general. However, a consequence of the guarantees we prove is that untrusted code cannot jump into the first 8 protected bytes, so it's not possible to get there in the first place under the reasonable assumptions that execution doesn't start from there at boot, and trusted code doesn't jump there.

\subsection{Universal contract}

We formalize the aforementioned properties with the universal contract shown in \cref{fig:universal-contract}. The contract constrains the possible outcomes of executing an arbitrary untrusted instruction under the assumption of correct IPE configuration.

\begin{figure}
  \centering
  \[\startcstep
    \hoarem
    {\begin{aligned}
      &\begin{alignedat}{2}
        &\reg{PC} &&\ptor \mathit{pc\_old} \\
        {}\ast{} &\reg{MPUIPC0}     &&\ptor \mathit{ipectl} \\
        {}\ast{} &\reg{MPUIPSEGB1}  &&\ptor \mathit{segb1} \\
        {}\ast{} &\reg{MPUIPSEGB2}  &&\ptor \mathit{segb2}
      \end{alignedat} && \circdef{ipe-regs-in}
      \\[2ex]
      &\begin{aligned}
        &\ast \pred{accessible\_addresses}~\mathit{segb1}~\mathit{segb2} \\
        &\ast \pred{other\_registers}
      \end{aligned} && \circdef{other-res-in}
      \\[2ex]
      &\begin{aligned}
        &\ast \pred{ipe\_configured}~\mathit{ipectl}~\mathit{segb1}~\mathit{segb2} \\
        &\ast \pred{untrusted}~\mathit{segb1}~\mathit{segb2}~\mathit{pc\_old}
      \end{aligned} && \circdef{assumptions}~
    \end{aligned}}
    {\texttt{execute(\(\mathit{instr}\))}}
    {\begin{aligned}
      &\phantom{\ast{}}
        \bigl(\exists\mathit{pc\_new}. \\
      &\quad\begin{aligned}
        &\reg{PC} \ptor \mathit{pc\_new} \\
        {}\ast{} &\pred{untrusted\_or\_entry\_point}
                ~\mathit{segb1}~\mathit{segb2}~\mathit{pc\_new}\bigr)
      \end{aligned} && \circdef{pc-new}
      \\[2ex]
      &\begin{alignedat}{2}
        &\ast \reg{MPUIPC0}     &&\ptor \mathit{ipectl} \\
        &\ast \reg{MPUIPSEGB1}  &&\ptor \mathit{segb1} \\
        &\ast \reg{MPUIPSEGB2}  &&\ptor \mathit{segb2}
      \end{alignedat} && \circdef{ipe-regs-out}
      \\[2ex]
      &\begin{aligned}
        &\ast \pred{accessible\_addresses}~\mathit{segb1}~\mathit{segb2} \\
        &\ast \pred{other\_registers}
      \end{aligned} && \circdef{other-res-out}
    \end{aligned}}\]
\caption{Universal contract for IPE security.}
\label{fig:universal-contract}
\end{figure}

In the precondition we require ownership of:
\begin{itemize}
\item the PC and IPE registers, whose values are bound to \(\mathit{pc\_old}\), \(\mathit{ipectl}\), \(\mathit{segb1}\) and \(\mathit{segb2}\) \circref{ipe-regs-in}, and
\item the remaining registers and all unprotected memory locations \circref{other-res-in}.
\end{itemize}
and we asssume that IPE is enabled and locked, and that the PC points to untrusted code \circref{assumptions}.

The custom predicates referenced in the contract are defined in \cref{fig:aux-preds}. \pred{ipe\_configured} checks that the \regbit{MPUIPENA} and \regbit{MPUIPLOCK} bits of the control register are set, and that the lower boundary indeed comes before the upper boundary. \pred{untrusted} that the specified address is not in the IPE region. \pred{accessible\_addresses} gives ownership of all unprotected locations.\footnote{
For an address \(a\), \(\pred{untrusted}~\mathit{segb1}~\mathit{segb2}~a
\wand \exists v.\; a \ptom v\) gives ownership of \(a\) with an unknown value \(v\) if we can prove that \(a\) is not in the IPE segment.} Since we don't treat the first 8 protected bytes as untrusted, we don't have a distinct \pred{unprotected} and use \pred{untrusted} instead.

% \pred{untrusted\_or\_entry\_point}

\begin{figure}
  \centering
  \begin{align*}
    &\begin{alignedat}{2}
      &\pred{ipe\_configured}~\mathit{ipectl}~\mathit{segb1}~\mathit{segb2}
      &&\triangleq \mathit{ipectl}[6] = 1 \wedge \mathit{ipectl}[7] = 1 \wedge \mathit{segb1} < \mathit{segb2}
      \\
      &\pred{untrusted}~\mathit{segb1}~\mathit{segb2}~\mathit{addr}
      &&\triangleq \mathit{addr} < (\mathit{segb1} \ll 4) \vee (\mathit{segb2} \ll 4) \leq addr
      \\
      &\pred{accessible\_addresses}~\mathit{segb1}~\mathit{segb2}
      &&\triangleq \bigast_{\mathclap{a \in \pred{Addr}}}
         \left(\pred{untrusted}~\mathit{segb1}~\mathit{segb2}~a
         \wand \exists v.\; a \ptom v\right) \\
      &\pred{other\_registers}
      &&\triangleq \bigast_{r \in \pred{Reg} \setminus \left\{ \reg{PC}, \reg{MPUIPC0}, \reg{MPUIPSEGB1}, \reg{MPUIPSEGB2}\right\}} \exists v.\; r \ptor v
    \end{alignedat} \\
    &\begin{aligned}
      \pred{untrusted\_or\_entry\_point}~\mathit{segb1}~\mathit{segb2}~\mathit{addr} \triangleq {}
      \pred{untrusted}~\mathit{segb1}~\mathit{segb2}~\mathit{addr}
      \vee \mathit{addr} = (\mathit{segb1} \ll 4 + 8)
    \end{aligned}
  \end{align*}
  \caption{Auxiliary predicates.}
  \label{fig:aux-preds}
\end{figure}

Without even looking at the postcondition, we know that if the contract is provable then \sail{execute} doesn't give read/write access to protected locations under these assumptions: by the proof rules of separation logic, to prove a triple about code that manipulates memory we need to assert ownership of all the locations the code could access, and here \sail{execute} only gets chunks for unprotected locations in its precondition.

The other security guarantees have to be spelled out in the postondition:
\begin{itemize}
\item the program counter may take a new value, that must be either outside the IPE boundaries or its entry point \circref{pc-new}, and
\item the IPE registers' values are the same as before executing the instruction \circref{ipe-regs-out}.
\end{itemize}
We also get back ownership of unprotected memory and all other registers \circref{other-res-out}, whose contents are allowed to have changed thanks to the existentials in the definitions of \pred{accessible\_addresses} and \pred{other\_regs}. This has no implications in terms of security, but is useful to keep being able to reason about Sail code after the call to \sail{execute}.

\section{Universal contract in Katamaran}

\Cref{lst:universal-contract} shows the translation of the universal contract to a Rocq definition to be verified by Katamaran. The contract is represented by a record with five fields (here listed without their \coq{sep_contract_} prefix):
\begin{labeling}{\coq{logic_variables}}
\item[\coq{logic_variables}] A list of the universally quantified variables used in the contract and their types. Up until now we treated free variables as universally quantified, but in Rocq we need to be explicit about them.
\item[\coq{localstore}] A list of variables bound to the parameters of the function.
\item[\coq{precondition}] The precondition, expressed in Katamaran's separation logic.

\item[\coq{result}] Variable that will be bound to the function's return value in the postcondition.
\item[\coq{postcondition}] The postcondition.
\end{labeling}

The only change from \cref{fig:universal-contract} is that we have a separate predicate \coq{asn_mpu_registers} for asserting ownership of the MPU registers, which are consequently not included in \coq{asn_other_registers}; this will be useful later.

\begin{listing}
  \begin{minted}{coq}
    Definition sep_contract_execute : SepContractFun execute :=
    {|
      sep_contract_logic_variables :=
        [ "ipectl" :: ty.wordBits; "segb1" :: ty.wordBits; "segb2" :: ty.wordBits
        ; "pc_old" :: ty.wordBits; "instr" :: ty.union Uast ];

      sep_contract_localstore := [term_var "instr"];

      sep_contract_precondition :=
          PC_reg         ↦ term_var "pc_old"
        ∗ MPUIPC0_reg    ↦ term_var "ipectl"
        ∗ MPUIPSEGB1_reg ↦ term_var "segb1"
        ∗ MPUIPSEGB2_reg ↦ term_var "segb2"

        ∗ asn_accessible_addresses "segb1" "segb2"
        ∗ asn_mpu_registers
        ∗ asn_other_registers

        ∗ asn_ipe_configured (term_var "ipectl") (term_var "segb1") (term_var "segb2")
        ∗ asn_untrusted (term_var "segb1") (term_var "segb2") (term_var "pc_old");

      sep_contract_result        := "u";
      sep_contract_postcondition :=
          term_var "u" = term_val ty.unit tt

        ∗ ∃ "pc_new",
          (PC_reg ↦ term_var "pc_new"
           ∗ asn_untrusted_or_entry_point
               (term_var "segb1") (term_var "segb2") (term_var "pc_new"))

        ∗ MPUIPC0_reg    ↦ term_var "ipectl"
        ∗ MPUIPSEGB1_reg ↦ term_var "segb1"
        ∗ MPUIPSEGB2_reg ↦ term_var "segb2"

        ∗ asn_accessible_addresses "segb1" "segb2"
        ∗ asn_mpu_registers
        ∗ asn_other_registers;
    |}.
  \end{minted}
  \caption{Universal contract in Katamaran.}
  \label{lst:universal-contract}
\end{listing}

There is however some nuance in the way the auxiliary predicates are defined. \coq{asn_mpu_registers} (\cref{lst:asn_mpu_registers}), \coq{asn_other_registers}, \coq{asn_ipe_configured} (\cref{lst:asn_ipe_configured}), \coq{asn_untrusted_or_entry_point} are simple Rocq definitions. On the other hand, \coq{asn_untrusted} and \coq{asn_accessible_addresses} are a Katamaran \coq{PurePredicate} and \coq{Predicate}, respectively.

\begin{listing}
  \begin{minted}{coq}
    Definition asn_mpu_registers {Σ} : Assertion Σ :=
        ∃ "MPUCTL0_reg"  , MPUCTL0_reg  ↦ term_var "MPUCTL0_reg"
      ∗ ∃ "MPUCTL1_reg"  , MPUCTL1_reg  ↦ term_var "MPUCTL1_reg"
      ∗ ∃ "MPUSEGB2_reg" , MPUSEGB2_reg ↦ term_var "MPUSEGB2_reg"
      ∗ ∃ "MPUSEGB1_reg" , MPUSEGB1_reg ↦ term_var "MPUSEGB1_reg"
      ∗ ∃ "MPUSAM_reg"   , MPUSAM_reg   ↦ term_var "MPUSAM_reg".
  \end{minted}
  \caption{Definition of \coq{asn_mpu_registers}.}
  \label{lst:asn_mpu_registers}
\end{listing}

\begin{listing}
  \begin{minted}{coq}
    Definition asn_ipe_configured {Σ} (ipectl segb1 segb2 : Term Σ ty.wordBits) :=
      asn.formula
        (formula_and
          (formula_and
             (* enabled *)
             (formula_relop bop.eq
                (term_vector_subrange 6 1 ipectl)
                (term_val (ty.bvec 1) [bv 0x1]))
             (* locked *)
             (formula_relop bop.eq
                (term_vector_subrange 7 1 ipectl)
                (term_val (ty.bvec 1) [bv 0x1])))
          (* segb1 < segb2 *)
          (formula_relop bop.lt
             (term_unsigned segb1)
             (term_unsigned segb2))).
  \end{minted}
  \caption{Definition of \coq{asn_ipe_configured}.}
  \label{lst:asn_ipe_configured}
\end{listing}

\subsection{Pure predicates and \texttt{asn\_untrusted}}

\emph{TODO please someone fact-check this :)}

Auxiliary predicates given as normal Rocq definitions are completely transparent to Katamaran: they are always expanded to their meaning. Complex preconditions that include disjunctions or many variables impact negatively the performance of the symbolic executor. Pure predicates are an answer to this: they are defined in terms of the usual assertion language, but must be unfolded manually by the user when needed. This way, the performance cost of the precondition is only paid when it is actually needed to solve a verification condition.

In our case, the eagerly-expanded version of \coq{asn_untrusted} proved to be problematic for the verification time: the disjunction (address before or after the IPE region) would cause Katamaran to needlessly explore both possibilities. Replacing it with a pure predicate, as shown in \cref{lst:pure-untrusted}, mitigates this issue.

We will see how the unfolding of \coq{asn_untrusted} is incorporated into the proofs in \cref{sec:TODO}.

\begin{listing}
  \begin{minted}{coq}
    (* Sig.v *)
    Inductive PurePredicate : Set :=
    | untrusted.

    Definition 𝑷_Ty (p : 𝑷) : Ctx Ty :=
      match p with
      | untrusted => [ty.Address; ty.Address; ty.Address]
      end.

    Definition puntrusted (segb1 segb2 pc : bv 16) : Prop :=
      bv.unsigned pc < bv.unsigned segb1 * 16 \/
      bv.unsigned segb2 * 16 <= bv.unsigned pc.

    Definition 𝑷_inst (p : 𝑷) : env.abstract Val (𝑷_Ty p) Prop :=
      match p with
      | untrusted => puntrusted
      end.

    (* Contracts.v *)
    Definition asn_untrusted {Σ} (segb1 segb2 pc : Term Σ ty.wordBits) : Assertion Σ :=
      asn.formula (formula_user untrusted [segb1; segb2; pc]).
  \end{minted}
  \caption{Definition of the pure predicate \coq{untrusted}.}
  \label{lst:pure-untrusted}
\end{listing}

\subsection{Spatial predicates and \texttt{asn\_accessible\_addresses}}
\label{sec:spatial-preds}

Katamaran's logic comes equipped with a predicate for asserting ownership of registers, which is derived automatically from the register definitions in the \usail specification; we used it already in \cref{lst:universal-contract}, \eg in \coq{PC_reg ↦ term_var "pc_old"}. Our universal contract deals with access to memory as well, so we need to extend the logic to be able to express ownership of memory locations. New resources can be added to Katamaran by defining a spatial predicate.

Spatial predicates may go beyond what is already expressible in the logic. Unlike pure predicates, they are not aliases for a logical assertion, but completely opaque objects whose meaning is entirely determined by the operations that the user defines on them. These operations come under the form of lemmas and foreign function contracts. (TODO iris model, maybe in background)

In our formalization we make use of a number of spatial predicates. We will discuss some now; the rest are relevant to the block verifier and will be presented in \cref{ch:block-verifier}.

\subsubsection{Memory location ownership with \coq{ptstomem}}

A predicate definition comprises of its name, argument types, and whether it is \emph{duplicable} and \emph{precise}. \Cref{lst:ptstomem} shows this done for the resource of memory locations, \coq{ptstomem}. We also define a notation \coq{l m↦ v} for easier use of the predicate and affinity with the built-in points-to for registers.

The \msp's memory is byte addressable, so the argument \coq{l} is an address (16 bits in our model) and \coq{v} a bitvector of length 8. Predicates that represent exclusive ownership of a resource, including \coq{ptstomem}, are not duplicable. This means that if a contract specifies a \coq{ptstomem} in its precondition, it \emph{consumes} the resource when it is invoked: the location will not be usable anymore unless it is returned in the postcondition.

A precise predicate partitions its arguments into inputs and outputs, and guarantees that from equality of inputs follows equality of outputs. \coq{ptstomem} is precise with the location as input and value as output: if we know that \coq{l m↦ v} and \coq{l m↦ u}, then \coq{v = u}, since the same memory location cannot store two different values at the same time. This information is used to improve the symbolic executor's performance.

\begin{listing}
  \begin{minted}{coq}
    (* Sig.v *)
    Inductive Predicate := ptstomem | (* ... *).

    Definition 𝑯_Ty (p : 𝑯) : Ctx Ty :=
      match p with
      | ptstomem => [ty.Address; ty.byteBits]
      | (* ... *)
      end.

    Definition 𝑯_precise (p : 𝑯) : option (Precise 𝑯_Ty p) :=
      match p with
      | ptstomem => Some (MkPrecise [ty.Address] [ty.byteBits] eq_refl)
      | (* ... *)
      end.

    (* Contracts.v *)
    Local Notation "l m↦ v" :=
      (asn.chunk (chunk_user ptstomem [l; v])) (at level 70).
  \end{minted}
  \caption{Definition of the \emph{points to memory} resource.}
  \label{lst:ptstomem}
\end{listing}

The utility of \coq{ptstomem} is established by the foreign function contracts for \sail{read_ram} and \sail{write_ram}, shown in \cref{lst:read-write-foreign-contracts}. Reading from and writing to memory requires ownership of the corresponding \coq{ptstomem} chunk, which is returned in the postcondition---unchanged when reading, with the updated value when writing. The return value of \sail{read_ram} is the value pointed to by the address in the precondition.

\begin{listing}
  \begin{minted}{coq}
    Definition sep_contract_read_ram : SepContractFunX read_ram :=
      {|
        sep_contract_logic_variables := ["l" :: ty.Address; "v" :: ty.byteBits];
        sep_contract_localstore      := [term_var "l"];
        sep_contract_precondition    := term_var "l" m↦ term_var "v";
        sep_contract_result          := "w";
        sep_contract_postcondition   := term_var "v" = term_var "w"
                                        ∗ term_var "l" m↦ term_var "v";
      |}.

    Definition sep_contract_write_ram : SepContractFunX write_ram :=
      {|
        sep_contract_logic_variables := ["l" :: ty.Address; "v" :: ty.byteBits; "w" :: ty.byteBits];
        sep_contract_localstore      := [term_var "l"; term_var "v"];
        sep_contract_precondition    := term_var "l" m↦ term_var "w";
        sep_contract_result          := "_";
        sep_contract_postcondition   := term_var "l" m↦ term_var "v";
      |}.
  \end{minted}
  % omitted term_var "u" = term_val ty.unit tt
  \caption{Contracts for the foreign functions \sail{read_ram} and \sail{write_ram}.}
  \label{lst:read-write-foreign-contracts}
\end{listing}

\subsubsection{Untrusted memory ownership with \coq{accessible_addresses}}

In our initial presentation of the universal contract, \sail{execute} received ownership of a subset of memory via \pred{accessible\_addresses} (\cref{fig:aux-preds}), defined by a separating conjunction over the set of untrusted addresses. Providing the symbolic executor with tens of thousands of \coq{ptstomem} chunks is unreasonable; an implicit representation of said separating conjunction is called for here.

The idea is to have a spatial predicate, \coq{accessible_addresses}, represent the whole untrusted memory. When we need to access a location, we extract the corresponding \coq{ptstomem} chunk from \coq{accessible_addresses}, transforming the latter into an \coq{accessible_addresses_without} predicate. This is necessary to avoid extracting the same \coq{ptstomem} multiple times, which would be unsound.\footnote{Remember that \coq{ptstomem} represents \emph{exclusive} ownership; in separation logic, \(\ell \ptom v \ast \ell \ptom v \to \bot\).} \coq{accessible_addresses_without} can later be recombined with the extracted chunk to regain full ownership of untrusted memory.

\Cref{lst:accessible_addresses} shows the definition of the two predicates in Katamaran. \coq{accessible_addresses} takes two parameters, the IPE boundaries; it will grant access to addresses outside of them. \coq{accessible_addresses_without} takes an additional parameter, the address that has been extracted. TODO precise with no outputs?

\begin{listing}
  \begin{minted}{coq}
    (* Sig.v *)
    Inductive Predicate :=
    | accessible_addresses
    | accessible_addresses_without
    | (* ... *).

    Definition 𝑯_Ty (p : 𝑯) : Ctx Ty :=
      match p with
      | accessible_addresses => [ty.wordBits; ty.wordBits]
      | accessible_addresses_without => [ty.wordBits; ty.wordBits; ty.Address]
      | (* ... *)
      end.

    Definition 𝑯_precise (p : 𝑯) : option (Precise 𝑯_Ty p) :=
      match p with
      | accessible_addresses => Some (MkPrecise [ty.wordBits; ty.wordBits] ε eq_refl)
      | accessible_addresses_without => Some (MkPrecise [ty.wordBits; ty.wordBits; ty.Address] ε eq_refl)
      | (* ... *)
      end.

    (* Contracts.v *)
    Local Notation asn_accessible_addresses segb1 segb2 :=
      asn.chunk_angelic (chunk_user accessible_addresses
                          [term_var segb1; term_var segb2]).
  \end{minted}
  \caption{Definition of \coq{accessible_addresses} and \coq{accessible_addresses_without}.}
  \label{lst:accessible_addresses}
\end{listing}

Extraction of a single location happens with the \coq{extract_accessible_ptsto} lemma shown in \cref{lst:accessible-lemmas}. The lemma takes an argument, \coq{addr}, which must be an untrusted address according to \coq{asn_untrusted}. Given an \coq{accessible_addresses} predicate matching the current IPE boundaries, the lemma returns a \coq{ptstomem} chunk for \coq{addr} and the corresponding \coq{accessible_addresses_without}. The companion lemma \coq{return_accessible_ptsto} recombines the two into the full \coq{accessible_addresses}.

\begin{listing}
  \begin{minted}{coq}
    Definition lemma_extract_accessible_ptsto : SepLemma extract_accessible_ptsto :=
      {|
        lemma_logic_variables :=
          ["addr" :: ty.Address; "segb1" :: ty.wordBits; "segb2" :: ty.wordBits];

        lemma_patterns := [term_var "addr"];

        lemma_precondition :=
            MPUIPSEGB1_reg ↦ term_var "segb1"
          ∗ MPUIPSEGB2_reg ↦ term_var "segb2"
          ∗ asn_accessible_addresses "segb1" "segb2"
          ∗ asn_untrusted (term_var "segb1") (term_var "segb2") (term_var "addr");

        lemma_postcondition :=
            MPUIPSEGB1_reg ↦ term_var "segb1"
          ∗ MPUIPSEGB2_reg ↦ term_var "segb2"
          ∗ asn_accessible_addresses_without "segb1" "segb2" "addr"
          ∗ ∃ "v", term_var "addr" m↦ term_var "v";
      |}.

    Definition lemma_return_accessible_ptsto : SepLemma return_accessible_ptsto :=
      {|
        lemma_logic_variables :=
          [ "addr" :: ty.Address; "segb1" :: ty.wordBits; "segb2" :: ty.wordBits ];

        lemma_patterns := [term_var "addr"];

        lemma_precondition :=
            asn_accessible_addresses_without "segb1" "segb2" "addr"
          ∗ ∃ "v", term_var "addr" m↦ term_var "v";

        lemma_postcondition := asn_accessible_addresses "segb1" "segb2";
      |}.
  \end{minted}
  \caption{Lemmas for the manipulation of \coq{accessible_addresses}}
  \label{lst:accessible-lemmas}
\end{listing}

These lemmas must be called explicitly in the \usail specification before and after each call to \sail{read_ram} and \sail{write_ram}, which require the \coq{ptstomem} chunk. In practice, this means adding \coq{stm_lemma} statements to the translation of the \sail{readMem} and \sail{writeMem} wrappers, as exemplified in \cref{lst:accessible-lemma-use}.

\begin{listing}
  \begin{minted}{coq}
    (stm_seq
      (stm_lemma extract_accessible_ptsto [exp_var "addr"])
      (stm_let "res" ty.byteBits
         (stm_foreign read_ram [exp_var "addr"])
         (stm_seq
            (stm_lemma return_accessible_ptsto [exp_var "addr"])
            (stm_exp (exp_var "res")))))
  \end{minted}
  \caption{Excerpt from the \usail definition of \sail{readMem}, showing the application of \coq{extract_accessible_ptsto} and \coq{return_accessible_ptsto}.}
  \label{lst:accessible-lemma-use}
\end{listing}

\section{Universal contract verification}
\label{sec:uc-verification}


With the aid of Katamaran, the universal contract can be proved without having to reason explicitly about every single statement in the \usail specification. Katamaran symbolically executes the object code of the contract starting from the user-provided precondition, and tries to match the generated postcondition with the desired one. If this process cannot be carried out automatically, the user is left with some verification conditions to prove manually, that typically involve non-trivial arithmetic.

We cannot expect Katamaran to process the universal contract in one go. Breaking down the verification goal into smaller, easier parts is fundamental in keeping the execution time of the solver reasonable. We do this in two ways:
\begin{itemize}
\item We define contracts for complex functions that are called multiple times. Once a contract for a function is defined, it will be used in place of symbolically executing the body for each call.
\item For functions that pattern match on their inputs, such as \sail{execute}, we can restrict the arguments to be a specific variant (or a limited subset) of the union or enum. We can then either prove the contract separately for all choices of inputs, or select a few representative cases.
\end{itemize}

In this section we present our proof of the universal contract for the \msp in a top-down fashion, starting from the security guarantee we ended up verifying and progressively showing the auxiliary specifications we had to prove to get there.

\subsection{Simplified statement of the universal contract}
\label{sec:uc-simple}

We proved the universal contract for a restricted set of instructions, which we deemed the most likely to be ill-specified. Specifically, we considered the move instruction with the four addressing modes supported by the Sail model (register, indexed, indirect register, indirect autoincrement), the call instruction, and all jump instructions. The \msp has no dedicated load/store instructions, relying on the addressing modes for memory accesses.

In practice, we defined three different contracts, each a copy of \cref{lst:universal-contract} with an additional precondition constraining the \sail{instr} argument to \sail{execute}. For move, this would be:
\begin{minted}{coq}
  term_var "instr" =
    term_union Uast Kdoubleop
      (term_tuple
         [term_val (ty.enum Edoubleop) MOV;
          term_var "bw";
          term_var "src_reg";
          term_var "src_am";
          term_var "dest_reg";
          term_var "dest_am"]);
\end{minted}

Additionally, since most of the 16 registers of the \msp are general purpose registers with no special behavior, we restrict the choice of \coq{src_reg} and \coq{dest_reg} to a few representative cases:
\begin{minted}{coq}
  ∗ asn_is_sample_reg (term_var "src_reg")
  ∗ asn_is_sample_reg (term_var "dest_reg")
\end{minted}
with:
\begin{minted}{coq}
  Definition asn_is_sample_reg {Σ} (reg : Term Σ (ty.enum Eregister)) : Assertion Σ :=
      reg = term_val (ty.enum Eregister) PC
    ∨ reg = term_val (ty.enum Eregister) SRCG1
    ∨ reg = term_val (ty.enum Eregister) R4.
\end{minted}
and we replace \coq{asn_other_registers} with \coq{asn_own_sample_regs}, asserting ownership of \reg{SRCG1} and \reg{R4} with arbitrary values (an existential variable), as well as some registers (\eg the stack pointer \reg{SP}) that are required by many functions. We don't include the program counter in \coq{asn_own_sample_regs} since we often need to bind its value to a variable for later use; quantifying it existentially would not be enough.

Clearly, this assumption weakens the security result. Faults in the specification that only appear with registers or instructions not covered here would not be identified, but we think this is very unlikely and preferred to focus on other aspects of the verification instead. Nonetheless, lifting the restriction on registers without running into performance issues would be possible by employing a predicate in the same vein as \coq{asn_accessible_addresses}, and verifying more instructions is just a matter of defining and proving more contracts.

\subsection{Contracts for reading from and writing to registers}

Due to the various addressing modes and the IPE locking mechanism, manipulating registers requires more than a simple read or assignment. Symbolic execution of \sail{readReg} and \sail{writeReg} each of the numerous times they are called leads to an explosion in the verification time.

To tackle this issue, we define and prove a contract for the auxiliary functions that do the heavy lifting of \sail{readReg} and \sail{writeReg}, one for each addressing mode they support.

\Cref{lst:read_register} shows the contract for the \sail{read_register}\footnote{Renamed \coq{read_am_register} in the \usail specification due to a name clash with a function provided by Katamaran.} function. The arguments are the register to read and an element of the \sail{BW} enum, which is either \sail{BYTE_INSTRUCTION} (meaning that we want to read the least significant byte of the register), or \sail{WORD_INSTRUCTION} (the whole 16-bit value); the return value, a variant of the union \sail{WordByte}, is either \sail{Byte(...)} or \sail{Word(...)} accordingly. This pattern shows up in many places in the \msp \usail specification.\footnote{A more sensible way to handle byte vs word instructions would be to zero-extend all bytes to words; this would reduce the number of variables in the contract, thus improving performance, and is more likely to be what happens in the hardware anyway. Still, it is interesting to evaluate how well Katamaran fares in the presence of this kind of quirks in existing specifications.} In the universal contract's verification, we never care about the specific value returned by \sail{read_register}, except for word reads on the \reg{PC} register. Consequently, in the postcondition the return value is an existentially quantified variable, which is asserted to be equal to the PC if appropriate.

\begin{listing}
  \begin{minted}{coq}
    Definition sep_contract_read_register : SepContractFun read_am_register :=
      {|
        sep_contract_logic_variables := [ ... ];
        sep_contract_localstore := [term_var "bw"; term_var "reg"];

        sep_contract_precondition :=
            PC_reg ↦ term_var "pc"
          ∗ asn_own_sample_regs
          ∗ asn_is_sample_reg (term_var "reg");

        sep_contract_result          := "v";
        sep_contract_postcondition   :=
            ( ∃ "b", (term_var "v" = term_union Uwordbyte Kbyte (term_var "b"))
            ∨ ∃ "w", (term_var "v" = term_union Uwordbyte Kword (term_var "w")
                      ∗ (term_var "reg" <> term_enum Eregister PC
                         ∨ term_var "w" = term_var "pc")))

          ∗ PC_reg ↦ term_var "pc"
          ∗ asn_own_sample_regs;
      |}.
  \end{minted}
  \caption{Contract for register-mode reads.}
  \label{lst:read_register}
\end{listing}

\sail{write_register} has an additional complexity: since it can modify the PC, which amounts to a jump, it needs to ensure that the change is allowed by the current IPE configuration. This is reflected in the contract (\cref{lst:write_register}), where we assert ownership of the IPE registers in the precondition, assuming that IPE is enabled and locked and that we are executing untrusted code. We don't need to keep track of the values written to the registers, so we return them as points-to chunks with existentially quantified variables (through \coq{asn_own_sample_regs}). Again, the exception is the PC, for which we have a few cases:
\begin{itemize}
\item Unchanged from the precondition \circref{write-reg-pc-old}.
\item If the write was to \reg{PC}, a new (unknown) value that is either in the untrusted region or the IPE entry point, since the write was performed by untrusted code \circref{write-reg-pc-new}. This guarantees that \sail{write_register} cannot be used to perform unauthorized jumps into the IPE region.
\item For word writes to \reg{PC} we also keep track of the value that was written \circref{write-reg-special}, as this information is necessary in the proof of \sail{read_autoincrement}'s contract.
\end{itemize}

\begin{listing}
  \startcstep
  \begin{minted}[escapeinside=??]{coq}
    Definition sep_contract_write_register : SepContractFun write_am_register :=
      {|
        sep_contract_logic_variables := [ ... ];
        sep_contract_localstore := [term_var "bw"; term_var "reg"; term_var "v"];

        sep_contract_precondition :=
            PC_reg         ↦ term_var "pc_old"
          ∗ MPUIPC0_reg    ↦ term_var "ipectl"
          ∗ MPUIPSEGB1_reg ↦ term_var "segb1"
          ∗ MPUIPSEGB2_reg ↦ term_var "segb2"

          ∗ asn_ipe_configured (term_var "ipectl") (term_var "segb1") (term_var "segb2")
          ∗ asn_untrusted (term_var "segb1") (term_var "segb2") (term_var "pc_old")
          ∗ asn_is_sample_reg (term_var "reg")

          ∗ asn_own_sample_regs;

        sep_contract_result        := "u";
        sep_contract_postcondition :=
            term_var "u" = term_val ty.unit tt

          ∗ (PC_reg ↦ term_var "pc_old" ?$\circdef{write-reg-pc-old}$?
             ∨ (term_var "reg" = term_val (ty.enum Eregister) PC
                ∗ ∃ "pc_new",
                  (PC_reg ↦ term_var "pc_new" ?$\circdef{write-reg-pc-new}$?
                   ∗ asn_untrusted_or_entry_point
                       (term_var "segb1") (term_var "segb2") (term_var "pc_new")

                   ∗ (term_var "bw" = term_enum Ebw WORD_INSTRUCTION
                      ∗ (term_var "v" = term_union Uwordbyte Kword (term_var "pc_new") ?$\circdef{write-reg-special}$?
                         ∨ ∃ "b", term_var "v" = term_union Uwordbyte Kbyte (term_var "b"))
                      ∨ term_var "bw" = term_enum Ebw BYTE_INSTRUCTION))))

          ∗ MPUIPC0_reg    ↦ term_var "ipectl"
          ∗ MPUIPSEGB1_reg ↦ term_var "segb1"
          ∗ MPUIPSEGB2_reg ↦ term_var "segb2"

          ∗ asn_own_sample_regs;
      |}.
  \end{minted}
  \caption{Contract for register-mode writes.}
  \label{lst:write_register}
\end{listing}

As for the other addressing modes, the contracts for \sail{read_indexed}, \sail{read_indirect} and \sail{read_autoincrement} are almost the same; \cref{lst:read_indexed} shows \sail{read_indexed}'s. These addressing modes work as follows:
\begin{description}
\item[indexed] Interpret the register as a base address. Fetch the next word after the instruction (the offset). Add it to the base address and return the value stored in memory at that location.
\item[indirect register] Return the value stored at the address contained in the register (like indexed without the offset).
\item[indirect autoincrement] Like indirect register mode, but increment the register by 1 or 2 (byte/word instruction) after the read is performed.
\end{description}

In the precondition, we assert ownership of the (untrusted) PC, the (properly configured) IPE registers, the sample registers and, new from before, untrusted memory via \coq{asn_accessible_addresses}. From the postcondition we get back those resources; in particular, the IPE registers will have the same value and the PC, though changed\footnote{We could be more precise here and state that \coq{pc_new} is \coq{pc_old + 2}, but we would need to add an exception for the \reg{CG2} register, for which the index is not used and thus the PC not incremented, and we don't need this information for the proof of the universal contract anyway.}, will still be in the untrusted region. Proving this contract also guarantees that indexed-mode reads from untrusted code cannot access protected memory, since we don't give access to it in the precondition.

\begin{listing}
  \begin{minted}{coq}
    Definition sep_contract_read_indexed : SepContractFun read_indexed :=
      {|
        sep_contract_logic_variables := [ ... ];
        sep_contract_localstore := [term_var "bw"; term_var "reg"];

        sep_contract_precondition :=
            PC_reg         ↦ term_var "pc_old"
          ∗ MPUIPC0_reg    ↦ term_var "ipectl"
          ∗ MPUIPSEGB1_reg ↦ term_var "segb1"
          ∗ MPUIPSEGB2_reg ↦ term_var "segb2"

          ∗ asn_ipe_configured (term_var "ipectl") (term_var "segb1") (term_var "segb2")
          ∗ asn_untrusted (term_var "segb1") (term_var "segb2") (term_var "pc_old")
          ∗ asn_is_sample_reg (term_var "reg")

          ∗ asn_accessible_addresses "segb1" "segb2"
          ∗ asn_mpu_registers
          ∗ asn_own_sample_regs;

        sep_contract_result        := "v";
        sep_contract_postcondition :=
            ∃ "pc_new",
              ( PC_reg ↦ term_var "pc_new"
              ∗ asn_untrusted
                  (term_var "segb1") (term_var "segb2") (term_var "pc_new"))

          ∗ MPUIPC0_reg    ↦ term_var "ipectl"
          ∗ MPUIPSEGB1_reg ↦ term_var "segb1"
          ∗ MPUIPSEGB2_reg ↦ term_var "segb2"

          ∗ asn_accessible_addresses "segb1" "segb2"
          ∗ asn_mpu_registers
          ∗ asn_own_sample_regs;
      |}.
  \end{minted}
  \caption{Contract for indexed-mode reads.}
  \label{lst:read_indexed}
\end{listing}

Asserting that the new PC is still untrusted is instrumental in composing contracts when multiple auxiliary functions are called to execute a single instruction. For example, a move with indexed-mode source and register-mode destination will first invoke \sail{read_indexed} and then \sail{write_register}, the latter requiring that the PC be untrusted. We could alternatively extend \sail{write_register}'s to also be applicable when executing IPE code, but this is a significant complication that doesn't net us any gain as far as the security property we prove is concerned.

It is not obvious that the update performed by \sail{read_indexed} actually keeps the PC in the untrusted region. This is true by a combination of our approximation of \coq{asn_untrusted}, that treats the first 8 bytes of the IPE region as trusted (see \cref{sec:target-properties}), and the fact that the 9th byte is the only allowed entry point.\footnote{Meaning that an exception is thrown if the PC enters the IPE region at any other address, and the postcondition is required to hold only for normally terminating programs, so the contract is satisfied in those cases.} It is then impossible to get to the entry point \(\reg{MPUIPSEGB1} \ll 4 + 8\) by adding 2 to the PC, which is either less than \(\reg{MPUIPSEGB1} \ll 4\) or at least \(\reg{MPUIPSEGB2} \ll 4\) with \(\reg{MPUIPSEGB1} < \reg{MPUIPSEGB2}\). Without the approximation it would be possible to start execution from \(\reg{MPUIPSEGB1} \ll 4 + 6\) and reach the entry point after the indexed read, forcing us to extend \sail{write_register}'s contract as mentioned above.
% figure?

It should be noted that \sail{write_register}'s postcondition does not guarantee that the PC is untrusted---it can't, since the user can set it to an arbitrary value, including the IPE entry point. This is not problematic: within the execution of a single instruction, \sail{write_register} is always the last auxiliary function with a contract to be called, so we don't need to satisfy any other precondition involving PC being untrusted after that.

Moving on, \sail{read_autoincrement}'s contract is the same as \sail{read_register}'s. An auto\-increment-\-mode read from PC increments it by 1 or 2, so the same argument holds for the resulting PC being untrusted. As \sail{read_indirect} doesn't modify any register, its postcondition states that the PC is unchanged from the precondition, and is otherwise the same as \sail{read_register}'s and \sail{read_autoincrement}'s.

\sail{write_indexed}'s contract is identical to \sail{read_indirect}'s, \ie like \cref{lst:read_indexed} but with \coq{PC_reg ↦ term_var "pc_old"} in the postcondition. One would expect the PC to be incremented by 2 as part of every indexed-mode write, but this happens elsewhere---this inconsistency with respect to \sail{read_indexed} comes from the original Sail specification.

\subsubsection{Proofs}
\label{sec:read-autoincrement-proof}

The above contracts can be proved automatically by Katamaran's symbolic executor, with the help of the contracts defined in the next sections and with the exception of \coq{read_autoincrement}'s. The verification condition we are left with requires proving that, assuming that \(\mathit{pc}\) is untrusted and \(\mathit{pc} + 2\) is the IPE entry point, then \(\mathit{pc} + 2\) is untrusted (and likewise for \(\mathit{pc} + 1\)). The conclusion is false (the entry point \emph{is} trusted), so we show that the premises are contradictory. Rocq's linear integer arithmetic solver (the \coq{lia} tactic) can handle the refutation, provided we first:
\begin{itemize}
\item unfold the definition of the \coq{puntrusted} predicate;
\item make the bounds due to the PC being a 16-bit bitvector explicit, \ie \(0 \leq \reg{PC} < 2^{16}\);
\item split the case where \(\mathit{pc} + 2\) overflows from the one where it doesn't.
\end{itemize}
The resulting proof script is shown in \cref{lst:autoincrement-proof}.

\begin{listing}
  \begin{minted}{coq}
    Lemma valid_contract_read_autoincrement : Symbolic.ValidContractWithFuel 10
        sep_contract_read_autoincrement fun_read_autoincrement.
    Proof.
      symbolic_simpl.
      repeat split; subst.
      - exfalso. unfold puntrusted in *. destruct (bv.unsigned_bounds v).
        destruct (bv.unsigned_add_view [bv [16] 0x2] v); cbn in *; lia.
      - exfalso. unfold puntrusted in *. destruct (bv.unsigned_bounds v).
        destruct (bv.unsigned_add_view [bv [16] 0x1] v); cbn in *; lia.
    Qed.
  \end{minted}
  \caption{Proof of \sail{read_autoincrement}'s contract.}
  \label{lst:autoincrement-proof}
\end{listing}

The implementation of \sail{read_autoincrement} updates the PC through \sail{write_register}, so here we are making use of the disjunct \circref{write-reg-special} of \sail{write_register}'s postcondition (\cref{lst:write_register}). Without that, we would ``forget'' (due to the existential quantification of \coq{pc_new}) that the value just written to \reg{PC} is \coq{pc_old} plus 1 or 2, and wouldn't have the information required to prove that it is untrusted as we did in \cref{lst:autoincrement-proof}.

\subsection{Contracts for memory read/write wrappers}

Another set of functions that are frequently used and have complex definitions are those that access memory with IPE permission checks. In particular, we provide contracts for:
\begin{itemize}
\item reading/writing to memory via \sail{readMem}, \sail{writeMem} and \sail{fetch};
\item modifying the memory-mapped MPU registers with \sail{write_mpu_reg_byte};
\item the function that performs the permission checks, \sail{check_byte_access}.
\end{itemize}

\Cref{lst:writeMem} shows \sail{writeMem}'s contract. We don't need to remember the actual values stored in memory for the purpose of verifying the universal contract, so the pre- and postcondition capture only the familiar security guarantees from the previous sections: the IPE registers don't change and protected memory locations are never accessed. \sail{readMem}'s contract is identical, with two exceptions: it doesn't have the \coq{v} parameter (the value being written), and the return value is not unit but an existentially quantified variable (the value of which we don't care about).

\begin{listing}
  % actually read_mem_aux
  \begin{minted}{coq}
    Definition sep_contract_writeMem : SepContractFun writeMem :=
      {|
        sep_contract_logic_variables := [ ... ];
        sep_contract_localstore := [term_var "bw"; term_var "addr"; term_var "v"];

        sep_contract_precondition :=
            PC_reg         ↦ term_var "pc"
          ∗ MPUIPC0_reg    ↦ term_var "ipectl"
          ∗ MPUIPSEGB1_reg ↦ term_var "segb1"
          ∗ MPUIPSEGB2_reg ↦ term_var "segb2"

          ∗ asn_ipe_configured (term_var "ipectl") (term_var "segb1") (term_var "segb2")
          ∗ asn_untrusted (term_var "segb1") (term_var "segb2") (term_var "pc")

          ∗ asn_accessible_addresses "segb1" "segb2"
          ∗ asn_mpu_registers;

        sep_contract_result        := "v";
        sep_contract_postcondition :=
            term_var "v" = term_val ty.unit tt

          ∗ PC_reg         ↦ term_var "pc"
          ∗ MPUIPC0_reg    ↦ term_var "ipectl"
          ∗ MPUIPSEGB1_reg ↦ term_var "segb1"
          ∗ MPUIPSEGB2_reg ↦ term_var "segb2"

          ∗ asn_accessible_addresses "segb1" "segb2"
          ∗ asn_mpu_registers;
      |}.
  \end{minted}
  \caption{Contract for writing to memory with permission checks.}
  \label{lst:writeMem}
\end{listing}

Ownership of the MPU registers is required as \sail{readMem} and \sail{writeMem} dispatch the operation to \sail{read_mpu_reg_byte} and \sail{write_mpu_reg_byte} if the address lies within the MPU register file range. \sail{write_mpu_reg_byte}'s contract is similar to \sail{writeMem}'s, but doesn't include the PC and \coq{asn_accessible_addresses} in the pre- and postcondition. It essentially asserts that the IPE registers are not modified.

The contract for \sail{fetch}, which reads the word pointed to by the PC and increments it by 2, differs from \sail{readMem} in what the postcondition asserts about the program counter:
\begin{minted}{coq}
  ∃ "pc_new",
    (PC_reg ↦ term_var "pc_new"
     ∗ asn_untrusted (term_var "segb1") (term_var "segb2") (term_var "pc_new"))
\end{minted}
Akin to \sail{read_indexed}, we are not interested in recording that \coq{pc_new} is \coq{pc_old + 2}, but we need to know that it is untrusted to be able to chain other contracts after a fetch.

\sail{check_byte_access} takes two parameters: the address that should be checked for access permission, and a flag that is true if the check is for a jump, \ie if the IPE entry point should be considered accessible. The function returns the unit value if the check passes, and throws an exception otherwise. The contract states that, assuming properly configured IPE registers and untrusted PC, on successful termination the provided address is either untrusted or (on jump target checks) the entry point:
\begin{minted}{coq}
  asn_untrusted (term_var "segb1") (term_var "segb2") (term_var "addr")
  ∨ asn.formula (formula_bool (term_var "jump"))
    ∗ asn_ipe_entry_point (term_var "segb1") (term_var "addr")
\end{minted}

\subsubsection{Proofs}

Symbolic execution of \sail{check_byte_access} leaves some arithmetic verification conditions, which can be proved as follows:
\begin{minted}{coq}
  Lemma valid_contract_check_byte_access : Symbolic.ValidContractWithFuel 10
      sep_contract_check_byte_access fun_check_byte_access.
  Proof.
    symbolic_simpl.
    repeat split; unfold puntrusted in *; try lia.
    congruence.
  Qed.
\end{minted}

\coq{write_mpu_reg_byte} also requires some manual intervention. Its verification condition involves proving that after the write the IPE registers are unchanged under contradictory assumptions (IPE registers both locked and not locked at the same time), and can be easily discharged via \coq{congruence}:
\begin{minted}{coq}
  Lemma valid_contract_write_mpu_reg_byte : Symbolic.ValidContractWithFuel 10
      sep_contract_write_mpu_reg_byte fun_write_mpu_reg_byte.
  Proof.
    symbolic_simpl.
    repeat split; congruence.
  Qed.
\end{minted}

\sail{readMem} and \sail{writeMem} need to obtain a \sail{ptstomem} chunk before calling \sail{read_ram} or \sail{write_ram} to perform the read or write. This happens by an invocation of the \sail{extract_accessible_ptsto} lemma which we need to add manually to the \usail specification as, detailed in \cref{sec:spatial-preds}. With that done, their contracts are verified automatically.

\subsection{Contracts for updating the program counter}

There are two auxiliary functions that are called in various places to update the PC: \sail{incPC}, incrementing it by 2, and \sail{setPC}. Their contracts have the usual assumptions about PC and IPE configuration. \sail{incPC}'s postcondition includes:
\begin{minted}{coq}
  ∃ "pc_new",
    (PC_reg ↦ term_var "pc_new"
     ∗ term_var "pc_new" = term_word_plus [bv 2] (term_var "pc_old")
     ∗ asn_untrusted (term_var "segb1") (term_var "segb2") (term_var "pc_new"))
\end{minted}
while \sail{setPC}'s:
\begin{minted}{coq}
  PC_reg ↦ term_var "pc_new"
  ∗ asn_untrusted_or_entry_point (term_var "segb1") (term_var "segb2") (term_var "pc_new");
\end{minted}
where \coq{pc_new} is \sail{setPC}'s argument. The key difference is that \sail{incPC} guarantees that the new PC is untrusted, while with \sail{setPC} it could be the entry point.

\subsubsection{Proofs}

\sail{setPC} is verified automatically. \sail{incPC} requires proving a verification condition resembling that of \sail{read_autoincrement}, and the proof is close to the one shown in \cref{lst:autoincrement-proof}.

Thanks to the use of \sail{incPC}, various functions that increment the PC (indexed-mode reads and writes, \sail{fetch}) can be handled entirely by the symbolic executor. \sail{read_autoincrement} isn't because it modifies the PC via \sail{write_register}.

\section{Summary}

Katamaran can prove automatically the security properties on move, jump and call instructions stated in \cref{sec:uc-simple}, provided we first reduce the complexity of the verification by introducing a number of supporting contracts. They are summarized in \cref{fig:deps}, which shows where each of them is used.

\begin{figure}
  \centering
  \begin{tikzpicture}[yscale=-1.2,xscale=1.1]
    \node (move)           at (-3.5, 0)   {\sail{execute} (move)};
    \node (call)           at (0, 0)      {\sail{execute} (call)};
    \node (jump)           at (3.5, 0)    {\sail{execute} (jump)};

    \node (readReg)        at (-2, 1.5)   {(\sail{readReg})};
    \node (writeReg)       at (2, 1.5)    {(\sail{writeReg})};
    \node (setPC)          at (7, 1.5)    {\sail{setPC}};

    \node (readRegister)   at (-5, 3)     {\sail{read_register}};
    \node (readIndexed)    at (-4, 3.5)   {\sail{read_indexed}};
    \node (readIndirect)   at (-2, 3)     {\sail{read_indirect}};
    \node (readAutoincr)   at (-0, 3.5)   {\sail{read_autoincr}.};
    \node (writeRegister)  at (3.5, 3.5)    {\sail{write_register}};
    \node (writeIndexed)   at (5.5, 3)      {\sail{write_indexed}};

    \node (fetch)          at (-4, 5)     {\sail{fetch}};

    \node (readMem)        at (-1, 6.5)   {\sail{readMem}};
    \node (writeMem)       at (3, 6.5)    {\sail{writeMem}};
    \node (incPC)          at (-4, 6.5)   {\sail{incPC}};

    \node (checkByte)      at (6.5, 9)      {\sail{check_byte_access}};
    \node (lemmaPtsTo)     at (-4, 8.7)   {\sail{extract_accessible_ptsto}\textsuperscript{\dag}};
    \node (lemmaPtsTo2)    at (-4, 9.1)   {\sail{return_accessible_ptsto}\textsuperscript{\dag}};

    \node (readRam)        at (-1, 10)    {\sail{read_ram}*};
    \node (writeRam)       at (1.5, 10)   {\sail{write_ram}*};
    \node (writeMpu)       at (4.5, 10)   {\sail{write_mpu_reg_byte}};

    \graph {
      (move) -> {(readReg),(writeReg)};
      (jump) -> {(setPC)};
      (call) -> {(readReg),(writeReg),(writeMem)};

      (readReg) -> {(readRegister),(readIndexed),(readIndirect),(readAutoincr)};
      (writeReg) -> {(writeRegister),(writeIndexed)};

      (readIndexed) -> {(readMem),(fetch)};
      (readIndirect) -> {(readMem)};
      (readAutoincr) -> {(readMem), (writeRegister)};
      (writeRegister) -> {(checkByte)};
      (writeIndexed) -> {(writeMem)};

      (fetch) -> {(readMem), (incPC)};

      (readMem) -> {(readRam),(checkByte),(lemmaPtsTo)};
      (writeMem) -> {(writeRam),(checkByte),(writeMpu),(lemmaPtsTo)};

      (incPC) -> {(checkByte)};
      (setPC) -> {(checkByte)};
    };
  \end{tikzpicture}
  \caption{Summary of the contracts defined in this chapter. An edge from \(f\) to \(g\) means that \(g\)'s contract is used in the verification of \(f\)'s. Functions in parentheses are included for illustrative purposes but do not have a contract. Lemmas and foreign functions are marked with \dag{} and * respectively.}
  \label{fig:deps}
\end{figure}
